\documentclass{article}
\usepackage{geometry}
\usepackage[namelimits,sumlimits]{amsmath}
\usepackage{amssymb,amsfonts}
\usepackage{multicol}
\usepackage{graphicx}
\usepackage[cm]{fullpage}
\newcommand{\tab}{\hspace*{5em}}
\newcommand{\conj}{\overline}
\newcommand{\dd}{\partial}
\newcommand{\ep}{\epsilon}
\newcommand{\openm}{\begin{pmatrix}}
\newcommand{\closem}{\end{pmatrix}}
\DeclareMathOperator{\cov}{cov}
\DeclareMathOperator{\var}{var}
\DeclareMathOperator{\rank}{rank}
\DeclareMathOperator{\im}{im}
\DeclareMathOperator{\Span}{span}
\DeclareMathOperator{\Null}{null}
\newcommand{\nc}{\newcommand}
\newcommand{\rn}{\mathbb{R}}
\nc{\cn}{\mathbb{C}}
\nc{\ssn}[1]{\subsubsection*{#1}}
\nc{\inner}[2]{\langle #1,#2\rangle}
\nc{\h}[1]{\widehat{#1}}
\nc{\tl}[1]{\widetilde{#1}}
\nc{\norm}[1]{\left\|{#1}\right\|}
\begin{document}

Name: Hall Liu

Date: \today 
\vspace{1.5cm}
\subsection*{1}
Over such a thin shell of thickness $\ep$, we can assume the density function to be roughly constant. Then, evaluating the density function at any point $x$ with norm $r$ gives
\[\frac{1}{(2\pi\sigma^2)^{d/2}}\exp\left(\frac{-1}{2}x^T\sigma^{-2}Ix\right)=\frac{1}{(2\pi\sigma^2)^{d/2}}\exp\left(\frac{-r^2}{2\sigma^2}\right)\]
The hypervolume of the surface of the sphere is $S_dr^{d-1}$ (since it's a $d-1$-dimensional set), so the volume of the shell is approximately $\ep S_dr^{d-1}$. The integral of the density over the shell is approximately the volume of the shell times the value of the density, or 
\[\ep\frac{S_dr^{d-1}}{(2\pi\sigma^2)^{d/2}}\exp\left(\frac{-r^2}{2\sigma^2}\right)\]
To find the maximum of $f$, differentiate it by $r$ to obtain
\[\frac{(d-1)S_dr^{d-2}}{(2\pi\sigma^2)^{d/2}}\exp\left(\frac{-r^2}{2\sigma^2}\right)-\frac{S_dr^d}{\sigma^2(2\pi\sigma^2)^{d/2}}\exp\left(\frac{-r^2}{2\sigma^2}\right)\]
Setting this equal to $0$ and canceling gives
\[0=(d-1)-\frac{r^2}{\sigma^2}\implies r=\pm\sigma\sqrt{d-1}\]
Now, to check that this is a maximum, taking the second derivative gives us a common, positive multiplier times the following:
\[(d-1)(d-2)-\frac{r^2}{\sigma^2}(2d-1)+\frac{r^4}{\sigma^4}\]
Plugging in the previously obtained value for the stationary point, we have
\[d^2-3d+2-(d-1)(2d-1)+(d-1)^2=-2d+2\]
which becomes negative for large $d$. Thus, the point is a maximum, and we can estimate it by $\sigma\sqrt{d}$ for large $d$.
%TODO: defn of p?
\subsection{2}
By the AM-GM inequality, we have $\frac{a+b}{2}\leq\sqrt{ab}$. Since the minimum of $a$ and $b$ is bounded above by the arithmetic mean, we thus also have $\min(a,b)\leq\sqrt{ab}$. 

Now, we have that $P(\text{error})=P(\text{error}|Y=1)P(Y=1)+P(\text{error}|Y=2)P(Y=2)$. If we let $\Sigma_i$ be the set which the classifier classifies as $\h{Y}=i$, then the above is equal to 
\[\int_{\Sigma_2}P(x|Y=1)P(Y=1)dx+\int_{\Sigma_1}P(x|Y=2)P(Y=2)dx\]
Over $\Sigma_1$, we have that $P(x|Y=1)P(Y=1)\leq P(x|Y=2)P(Y=2)$ by the form of the Bayes classifier, and vice versa for $\Sigma_2$. Thus, the integrals above are bounded above by
\begin{align*}
&\frac{1}{2}\int_{\Sigma_2}P(x|Y=1)P(Y=1)+P(x|Y=2)P(Y=2)dx+ \frac{1}{2}\int_{\Sigma_1}P(x|Y=1)P(Y=1)+P(x|Y=2)P(Y=2)dx\\
    &=\frac{1}{2}\int_{\rn^n}P(x|Y=1)P(Y=1)+P(x|Y=2)P(Y=2)dx
\end{align*}
if we replace the integrands with the average of the two integrands. Now, using the AM-GM inequality, we have that this is bounded above by 
\begin{align*}
\int_{\rn^n}\sqrt{P(x|Y=1)P(Y=1)P(x|Y=2)P(Y=2)}dx&=\sqrt{P(Y=1)P(Y=2)}\int_{\rn^n}\sqrt{P(x|Y=1)P(x|Y=2)}dx\\
&\leq\frac{1}{2}\int_{\rn^n}\sqrt{P(x|Y=1)P(x|Y=2)}dx
\end{align*}
by another application of AM-GM to $\sqrt{P(Y=1)P(Y=2)}$, noting that the average of $P(Y=1)$ and $P(Y=2)$ is $\frac{1}{2}$.
\subsection*{3}
\ssn{a}
The decision boundary is the point at which $P(x|Y=1)P(Y=1)=P(x|Y=2)P(Y=2)$, or $\pi_1\exp\left((x-\mu_1)^2/(2\sigma^2)\right)=\pi_2\exp\left((x-\mu_2)^2/(2\sigma^2)\right)$. Solving for $x$, we have 
\[x^*=\frac{2\sigma\log\frac{\pi_1}{\pi_2}+\mu_2^2-\mu_1^2}{2(\mu_2-\mu_1)}\]
\ssn{b}
The error probability is
\[\int_{-\infty}^{x^*}P(x|Y=2)P(Y=2)dx+\int_{x^*}^\infty P(x|Y=1)P(Y=1)dx=\pi_2\Phi\left(\frac{x^*-\mu_2}{\sigma}\right)+\pi_1-\pi_1\Phi\left(\frac{x^*-\mu_1}{\sigma}\right)\]

Rewrite $x^*-\mu_2=\frac{2\sigma\log\frac{\pi_1}{\pi_2}}{2(\mu_2-\mu_1)}+\frac{\mu_2+\mu_1}{2}-\mu_2$. Then, dividing this by $\sigma$ produces a term constant in $\sigma$ and a term that goes to $-\infty$ as $\sigma\to0$. Thus, the $\pi_2$ term in the error vanishes. Similarly, $\frac{x^*-\mu_1}{\sigma}$ goes to $\infty$ as $\sigma\to0$, which means that the overall error term approaches $\pi_1-\pi_1=0$.
\ssn{c}
As $\pi_1\to0$, the log term in the decision boundary approaches $-\infty$, dragging the decision boundary with it, as all other things are fixed. In this case, always classifying things as class 2 would produce an error rate of $\pi_1$, which is known to be small.
\ssn{d}
{
\nc{\lot}{L_{1,2}}
\nc{\lto}{L_{2,1}}
Let $S_1$ be the set of points in class $1$ and $S_2$ the same for class 2. If we predict class 1, the expected loss is $E_X(\lto P(Y=2|X)$. If we predict class 2, then it is $E_X(\lot P(Y=1|X))$. If we minimize this pointwise, we should predict class 1 when $\lto P(Y=2|X)\leq\lot P(Y=1|X)$, and class 2 otherwise.

Rewriting the conditional probabilities, the prediction is then $\text{argmax}\left(\lto P(X|Y=2)P(Y=2), \lot P(X|Y=1)P(Y=1)\right)$. Setting these equal to find the boundary, we have
\[\lto \pi_2\exp\left((x-\mu_2)^2/(2\sigma^2)\right)=\lot\pi_1\exp\left((x-\mu_1)^2/(2\sigma^2)\right)\]
Solving for $x$, we obtain
\[x^*=\frac{2\sigma\log\frac{\lot\pi_1}{\lto\pi_2}+\mu_2^2-\mu_1^2}{2(\mu_2-\mu_1)}\]
Now, if we choose $\lot$ to be proportional to $\pi_1^{-1}$ and choose $\lto$ to be proportional to $\pi_2^{-1}$ (possibly with different constants), we will avoid the degeneracy problem.
\ssn{e}
Using these values, we have that $x^*=1$, so plugging into the expression for the error rate from (b) gives $0.5\Phi(-1)+0.5-0.5\Phi(1)=0.1587$
}
\subsection*{4}
The expected loss of some decision function $\h{H}$ is $E_{X, Y}(L(\h{H}(x), Y)$. Writing out the integral gives
\[\int_{\rn^n}\sum_{k=1}^KL(\h{H}(x), k)p(x, k) dx=\int_{\rn^n}p(x)\sum_{k=1}^KL(\h{H}(x), k)P(k|x)dx\]

%If we define $R_i$ as the subset of $\rn^n$ for which $\h{H}(x)=i$, we can separate the integral as follows
%\[\sum_{i=1}^K\int_{R_i}p(x)\sum_{k=1}^KL(i, k)P(k|x)dx\]
%If there is some subset $A$ of $R_i$ for which there exists a $j$ such that  for all $x\in A$, then moving $A$ from $R_i$ to $R_j$ will reduce the value of the expected loss by comparison 

Define $R_i$ as $\{x\in\rn^n|\h{H}(x)=i\}$. Then, suppose that there exists some subset of nonzero measure $A$ of $R_i$ such that $\sum_{k=1}^KL(i, k)P(k|x)\ge\sum_{k=1}^KL(j, k)P(k|x)$. Decompose the above integral as
\[\int_{\rn^n-A}p(x)\sum_{k=1}^KL(\h{H}(x), k)P(k|x)dx+\int_Ap(x)\sum_{k=1}^KL(i, k)P(k|x)dx\]
Then, changing the class of $A$ from $i$ to $j$ will result in a strict decrease in the expected loss, which means that the optimal decision rule must satisfy, for each $i$, $\sum_{k=1}^KL(i, k)P(k|x)\ge\sum_{k=1}^KL(j, k)P(k|x)$ for all $j\neq i$ and almost all $x\in R_i$. The rule given satisfies this condition, and since there are a finite number of classes, changing the rule on a set of measure zero from the region associated with each class will not result in a change in the expected loss. Thus, the rule given is optimal.
\subsection*{5}
\ssn{a}
We have $\cov(X_2,X_2)=v_2$, so writing $X_2=\alpha X_1+Z$ gives
\[\cov(\alpha X_1+Z, \alpha X_1+Z)=\cov(\alpha X_1, \alpha X_1)+0+\cov(Z, Z)=\alpha^2v_1+\var(Z)=v_2\]
In addition, using $\cov(X_1, X_2)=a$ gives us
\[a=\cov(X_1, \alpha X_1+Z)=\alpha v_1+0\implies \alpha=\frac{a}{v_1}\]
Substituting into the first equation gives us $\var(Z)=v_2-\frac{a^2}{v_1}$

Finally, since $E(X_1)=E(X_2)=0$, we must also have $E(Z)=0$.
\ssn{b}
We know that $Z$ is Gaussian because we can express $Z$ as the sum of two Gaussians, $X_2-\alpha X_1$. Thus, it is independent from $X_1$ because their covariance is zero. The variance was derived above.
\ssn{c}
We have $E(X_2|X_1=x)=E(\alpha x+Z)=\alpha x$ and $\var(X_2|X_1=x)=\var(\alpha x+Z)=\var(Z)$. 
\ssn{d}
Since the samples are all independent, the cross terms in the formula for the variance of a sum disappear and we get
\[\var(\h{a})=\frac{1}{n^2}\sum_{i=1}^n\var(X_{i1}X_{i2})=\frac{1}{n}\var(X_1X_2)\]
Then, we have $\var(X_1X_2)=E((X_1X_2)^2)-E(X_1X_2)^2=E((X_1X_2)^2)-a^2$, since the means are zero.

To obtain the first term, note that it is a cross moment and can be obtained by applying $\frac{\partial^4}{\partial x_1^2\partial x_2^2}$ to the mgf of the distribution and evaluating at zero. The mgf itself is
\[\exp\left(\frac{1}{2}\openm x_1&x_2\closem\openm v_1&a\\a&v_2\closem\openm x_1\\x_2\closem\right)=\exp\left(\frac{1}{2}(x_1^2v_1+2x_1x_2a+x_2^2v_2)\right)\]

Differentiating this wrt $x_1$ twice and wrt $x_2$ once gives 
\[ax_1v_1T+x_1v_1(x_1v_1+x_2a)(x_2v_2+x_1a)T+v_1(x_2v_2+x_1a)T\]
where $T$ is the exponential term from above. Then, taking the derivatives and then evaluating gives $v_1v_2$, so the variance of $\h{a}$ is $\frac{1}{n}(v_1v_2-a^2)$.
\subsection*{6}
\end{document}
