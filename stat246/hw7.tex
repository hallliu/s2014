\documentclass{article}
\usepackage{geometry}
\usepackage[namelimits,sumlimits]{amsmath}
\usepackage{amssymb,amsfonts}
\usepackage{multicol}
\usepackage{graphicx}
\usepackage[cm]{fullpage}
\usepackage{mathrsfs}
\newcommand{\tab}{\hspace*{5em}}
\newcommand{\conj}{\overline}
\newcommand{\dd}{\partial}
\newcommand{\ep}{\epsilon}
\newcommand{\openm}{\begin{pmatrix}}
\newcommand{\closem}{\end{pmatrix}}
\DeclareMathOperator{\cov}{cov}
\DeclareMathOperator{\var}{var}
\DeclareMathOperator{\tr}{tr}
\DeclareMathOperator{\rank}{rank}
\DeclareMathOperator{\im}{im}
\DeclareMathOperator{\Span}{span}
\DeclareMathOperator{\Null}{null}
\newcommand{\nc}{\newcommand}
\newcommand{\rn}{\mathbb{R}}
\nc{\cn}{\mathbb{C}}
\nc{\ssn}[1]{\subsubsection*{#1}}
\nc{\inner}[2]{\langle #1,#2\rangle}
\nc{\h}[1]{\widehat{#1}}
\nc{\tl}[1]{\widetilde{#1}}
\nc{\norm}[1]{\left\|{#1}\right\|}
\begin{document}

Name: Hall Liu

Date: \today 
\vspace{1.5cm}

\subsection*{1}
\ssn{a}
Suppose $j$ is a neighbor of $i$. Then, there exists some clique that contains both $i$ and $j$, so $j$ is in the union of the cliques containing $i$. Conversely, suppose $j$ is in the union of cliques containing $i$. Then, $i$ and $j$ coexist in a clique together, so they have an edge between them, which means that $j$ is a neighbor of $i$.
\ssn{b}
Denote the set of cliques which contain $X_i$ by $C(i)$ and its complement in the set of cliques by $C(-i)$. Then, we have that $P(X_i|X_{N_i})=P(X_i|X_{-i})$ (where $X_{-i}$ is all nodes except $X_i$), which can be written as 
\[\frac{\exp\left(\beta\sum_l\psi_{C_l}(X_{C_l})\right)}{\exp\left(\beta\sum_{C\in C(-i)}\psi_C(X_C)\right)\sum_{X_i}\exp\left(\beta\sum_{C\in C(i)}\psi_C(X_C)\right)}=\frac{\exp\left(\beta\sum_{C\in C(i)}\psi_{C}(X_{C})\right)}{\sum_{X_i}\exp\left(\beta\sum_{C\in C(i)}\psi_C(X_C)\right)}\]
Using this, then, we can write the log-pseudolikelihood as
\[\sum_{n=1}^N\sum_{i=1}^d\left(\beta\sum_{C\in C(i)}\psi_{C}(X_{C})-\log\left(\sum_{X_i}\exp\left(\beta\sum_{C\in C(i)}\psi_C(X_C)\right)\right)\right)\]
where the summation over $n$ operates via an implicit superscript $(n)$ over each datum $X$. Differentiating wrt $\beta$ and setting to $0$ gives 
\[0=\sum_{n=1}^N\sum_{i=1}^d\left(\sum_{C\in C(i)}\psi_{C}(X_{C})-\frac{\sum_{X_i}\left(\sum_{C\in C(i)}\psi_C(X_C)\right)\exp\left(\beta\sum_{C\in C(i)}\psi_C(X_C)\right)}{\sum_{X_i}\exp\left(\beta\sum_{C\in C(i)}\psi_C(X_C)\right)}\right)\]
It doesn't look like this can be simplified any further without knowing the forms of the cliques.
\ssn{c}

\end{document}
