\documentclass{article}
\usepackage{geometry}
\usepackage[namelimits,sumlimits]{amsmath}
\usepackage{amssymb,amsfonts}
\usepackage{multicol}
\usepackage{graphicx}
\usepackage{mathtools}
\usepackage[cm]{fullpage}
\newcommand{\tab}{\hspace*{5em}}
\newcommand{\conj}{\overline}
\newcommand{\dd}{\partial}
\newcommand{\ep}{\epsilon}
\newcommand{\openm}{\begin{pmatrix}}
\newcommand{\closem}{\end{pmatrix}}
\DeclareMathOperator{\cov}{cov}
\DeclareMathOperator{\var}{var}
\DeclareMathOperator{\rank}{rank}
\DeclareMathOperator{\tr}{tr}
\DeclareMathOperator{\im}{im}
\DeclareMathOperator{\Span}{span}
\DeclareMathOperator{\Null}{null}
\newcommand{\nc}{\newcommand}
\newcommand{\rn}{\mathbb{R}}
\newcommand{\zn}{\mathbb{Z}}
\nc{\cn}{\mathbb{C}}
\nc{\ssn}[1]{\subsubsection*{#1}}
\nc{\inner}[2]{\langle #1,#2\rangle}
\nc{\h}[1]{\widehat{#1}}
\nc{\tl}[1]{\widetilde{#1}}
\nc{\norm}[1]{\left\|{#1}\right\|}
\DeclarePairedDelimiter\ceil{\lceil}{\rceil}
\DeclarePairedDelimiter\floor{\lfloor}{\rfloor}
\begin{document}
Name: Hall Liu

Date: \today 
\vspace{1.5cm}

\subsection*{1}
\ssn{a}
First, note that since $F$ is strictly increasing, we must have that $F(x)\in(0,1)$ for any $x\in\rn$ and $F^{-1}$ exists. Then, for any $y\in(0,1)$, we have $P(Y<y)=P(F(X)<y)=P(X<F^{-1}(y))=F(F^{-1}(y))=y$, which implies that $Y$ has a cdf of $0$ on $(-\infty, 0)$, $y$ on $[0,1]$, and $1$ on $(1, \infty)$. Thus, we have $Y\sim\text{Unif}(0, 1)$.

Going the other way, we have $P(X<x)=P(F^{-1}(U)<x)=P(U<F(x))=F(x)$.

\ssn{b}
The joint density $f_{X,Y}(x,y)$ of $X$ and $Y$ is the product of their densities (which is 1) since they are independent. Thus, for the difference, for $z\in(0,1]$, we have
\[P(Z<z)=P(X-Y<z)=P(X<z+Y)=\int_0^{1-z}\int_0^{z+y}1dxdy+\int_{1-z}^1\int_0^11dxdy=\int_0^{1-z}(z+y)dy+z=-\frac{z^2}{2}+z+\frac{1}{2}\]
For $z\in[-1,0]$, we have
\[P(Z<z)=P(X-Y<z)=P(X<z+Y)=\int_{-z}^1\int_0^{y+z}1dxdy=\int_{-z}^1(z+y)dy=\frac{z^2}{2}+z+\frac{1}{2}\]
Differentiating produces the density $z+1$ on $[-1,0]$ and $-z+1$ on $(0,1]$. 

For the minimum, we want $X<z$ or $Y<z$, which is equivalent to the negation of ($X\geq z$ and $Y\geq z$), so the integral for the cdf of $Z$ is 
\[1-\int_{1-z}^1\int_{1-z}^11dxdy=1-(1-z)^2\]
and differentiating this produces the density $2(1-z)$.
\ssn{c}
The density of $X$ is $f_X(x)=\frac{1}{\sqrt{2\pi}}e^{-\frac{x^2}{2}}$. Taking the expectation of $e^X$ wrt this density gives
\[\frac{1}{\sqrt{2\pi}}\int_{-\infty}^\infty\exp\left(-\frac{x^2}{2}+x\right)dx=\frac{1}{\sqrt{2\pi}}e^{1/2}\int_{-\infty}^\infty\exp\left(-\frac{(x-1)^2}{2}\right)=e^{1/2}\]
The variance can be calculated by $E(Y^2)-(E(Y))^2$, and evaluating $E(Y^2)=E(e^{2X})$ gives
\[\frac{1}{\sqrt{2\pi}}\int_{-\infty}^\infty\exp\left(-\frac{x^2}{2}+2x\right)dx=\frac{1}{\sqrt{2\pi}}e^2\int_{-\infty}^\infty\exp\left(-\frac{(x-2)^2}{2}\right)=e^2\]
so the variance is $e^2-e$
\ssn{d}
$\var_Y(Y|X)=E_Y(Y^2|X)-(E_Y(Y|X))^2$, so the first term is $E_X(\var_Y(Y|X))=E_X(E_Y(Y^2|X))-E_X((E_Y(Y|X))^2)=E_Y(Y^2)-E_X((E_Y(Y|X))^2)$. Similarly, the second term is $E_X((E_Y(Y|X))^2)-(E_X(E_Y(Y|X)))^2=E_X((E_Y(Y|X))^2)-(E_Y(Y))^2$. Adding them together cancels out the $E_X((E_Y(Y|X))^2)$ term, which results in $E_Y(Y^2)-(E_Y(Y))^2=\var(Y)$.
\subsection*{2}
\ssn{a}
Since $X^TX$ is nonsingular and $n>d$, this means that $X$ has rank $d$. The least-squares estimate $\h{\beta}$ satisfies $\h{\beta}=\text{argmin}_\beta\|X\beta-y\|_2^2$. Decompose $X$ via the SVD $X=U\Sigma V^T$. Then, we have 
\[\|X\beta-y\|_2^2=\norm{U\Sigma V^T\beta-y}_2^2=\norm{\Sigma \gamma-U^Ty}_2^2\]
where $\gamma=V^T\beta$, by the unitary invariance of the $2$-norm. Let $z=U^Ty$. We know that precisely the first $d$ entries in $\Sigma $ are nonzero because $X$ is rank $d$, so the norm expression is 
\[\sum_{i=1}^d(\sigma_i\gamma_i-z_i)^2+\sum_{i=d+1}^nz_i^2\]
To minimize this, we set $\gamma_i=z_i/\sigma_i$ for $i\in[1..d]$, and we can choose the others arbitrarily. Might as well make the others zero, so then we have that $\h{\beta}=V\h{\gamma}=V\Sigma^+U^Ty$, where $\Sigma^+$ is diagonal with entries $\frac{1}{\sigma_i}$ where $\sigma_i\neq0$ and zero elsewhere.

Now, note that both $V\Sigma^+U^TX$ and $(X^TX)^{-1}X^T$ satisfy the requirements for being the Moore-Penrose pseudoinverse of $X$. The calculations are a bit tedious to type out and not very illuminating, so I'll omit them. Thus, by the uniqueness of the pseudoinverse, they must be equal. Thus, we have $\h{\beta}=(X^TX)^{-1}X^Ty$.
\ssn{b}
$HX=X(X^TX)^{-1}X^TX=X$
\ssn{c}
$H^T=(X(X^TX)^{-1}X^T)^T=X(X^TX)^{-T}X^T=X(X^TX)^{-1}X^T$
\ssn{d}
$HH=X(X^TX)^{-1}X^TX(X^TX)^{-1}X^T=X(X^TX)^{-1}X^T=H$
\ssn{e}
By (d), $H$ is a projection matrix. Thus, we just need to show that the image of $H$ is the column space. Since the image of $X$ is the column space (a subset of $\rn^n$), this means that we need to show that $(X^TX)^{-1}X^T$ has rank $d$ (or that its image is all of $\rn^d$). This is true because $(X^TX)^{-1}$ is invertible and $X^T$ has rank $d$ also. 
\ssn{f}
Traces of products are invariant under cyclic permutation, so $\tr(H)=\tr(X(X^TX)^{-1}X^T)=\tr(X^TX(X^TX)^{-1})=\tr(I_d)=d$.
\subsection*{3}
\ssn{b}
I have a general policy against running unknown scripts on my computer, and since I know that I have a large part of the Python toolchain set up already, I decided to read the setup script (\verb|toolkit.sh|) and replicate the necessary parts. 

The first difficulty came when trying to install the packages. I ran into a syntax error while installing czipfile via pip, and this turned out to be because my system defaults to Python 3. Deleting the virtualenv and re-running mkvirtualenv with the option \verb|--python=/usr/bin/python2.7| fixed the issue.

The Python version problem came up again when creating the bin/notebook file. I modified the line to start with \verb|python2.7 /usr/bin/ipython| to force the script to use 
\end{document}
